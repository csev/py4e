% El contenido de este archivo tiene 
% Copyright (c) 2009-  Charles R. Severance, Todos los Derechos Reservados

% LATEXONLY

\input{latexonly}

\newtheorem{ex}{Ejercicio}[chapter]

\begin{latexonly}

\renewcommand{\blankpage}{\thispagestyle{empty} \quad \newpage}

\thispagestyle{empty}

\begin{flushright}
\vspace*{2.0in}

\begin{spacing}{3}
{\huge Python para informáticos}\\
{\Large Explorando la información}
\end{spacing}

\vspace{0.25in}

Version \theversion

\vspace{0.5in}


{\Large
Charles Severance\\
}

\vfill

\end{flushright}

%--copyright--------------------------------------------------
\pagebreak
\thispagestyle{empty}

{\small
Copyright \copyright ~2009- Charles Severance.

Traducción al español por Fernando Tardío.


Historial de impresiones:

\begin{description}
	
\item[Agosto 2015:] Primera edición en español de \emph{Python para Informáticos:
Explorando la información}.

\item[May 2015:] Permiso editorial gracia a Sue Blumenberg.

\item[Octubre 2013:] Revisión completa a los capítulos 13 y 14
para cambiar a JSON y usar OAuth.
Añadido capítulo nuevo en Visualización.

\item[Septiembre 2013:] Libro publicado en Amazon CreateSpace

\item[Enero 2010:] Libro publicado usando la máquina
Espresso Book de la Universidad de Michigan.

\item[Diciembre 2009:] Revisión completa de los capítulos 2-10 de
\emph{Think Python: How to Think Like
a Computer Scientist}
y escritura de los capítulos 1 y 11-15, para
producir 
\emph{Python for Informatics: Exploring the Information}

\item[Junio 2008:] Revisión completa, título cambiado por
\emph{Think Python: How to Think Like
a Computer Scientist}.

\item[Agosto 2007:] Revisión completa, título cambiado por
\emph{How to Think Like a (Python) Programmer}.

\item[Abril 2002:] Primera edición de \emph{How to Think Like
a Computer Scientist}.

\end{description}

\vspace{0.2in}

Este trabajo está licenciado bajo una licencia
Creative Common
Attribution-NonCommercial-ShareAlike 3.0 Unported.
Esta licencia está
disponible en
\url{creativecommons.org/licenses/by-nc-sa/3.0/}. Puedes
consultar qué es lo que el autor considera usos comerciales y no comerciales
de este material, así como las exenciones de licencia
en el apéndice titulado Detalles del Copyright.

Las fuentes \LaTeX\ de la versión 
\emph{Think Python: How to Think Like
	a Computer Scientist}
de este libro están disponibles en
\url{http://www.thinkpython.com}.

\vspace{0.2in}

} % end small

\end{latexonly}


% HTMLONLY

\begin{htmlonly}

% TITLE PAGE FOR HTML VERSION

{\Large \thetitle}

{\large 
Charles Severance}

Version \theversion

\setcounter{chapter}{-1}

\end{htmlonly}
