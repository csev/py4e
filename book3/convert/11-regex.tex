% The contents of this file is 
% Copyright (c) 2009-  Charles R. Severance, All Righs Reserved

\chapter{Expresiones regulares}

Hasta ahora hemos estado leyendo archivos, buscando patrones y extrayendo varios
fragmentos de líneas que encontrábamos interesantes. Hemos estado usando métodos de cadena, como {\tt split}
y {\tt find}, usando listas y rebanado de cadenas para extraer porciones de esas líneas.
\index{expresiones regulares}
\index{regex}
\index{re, módulo}

Esta tarea de buscar y extraer es tan común que Python tiene una librería muy potente
llamada {\bf expresiones regulares}, que se encarga de muchas de estas tareas de forma elegante. La
razón por la que no hemos introducido las expresiones regulares antes en este libro se debe a que,
a pesar de que son muy potentes, también son un poco complicadas y lleva algún tiempo acostumbrarse a su
sintaxis.

Las expresiones regulares tienen casi su propio pequeño lenguaje de programación para buscar y analizar
las cadenas. De hecho, se han escritos libros enteros sobre el tema de las expresiones regulares.
En este capítulo, nosotros sólo cubriremos lo más básico acerca de las expresiones regulares.
Para obtener más detalles sobre ellas, puedes consultar:

\url{http://es.wikipedia.org/wiki/Expresion_regular}

\url{https://docs.python.org/2/library/re.html}

La librería de expresiones regulares {\tt re}, debe ser importada en el programa antes de poder utilizarlas.
El uso más simple de esta librería es la función {\tt search()}. El programa siguiente
demuestra un uso trivial de la función {\tt search}.
\index{regex!search}

\beforeverb
\begin{verbatim}
import re
manf = open('mbox-short.txt')
for linea in manf:
    linea = linea.rstrip()
    if re.search('From:', linea) :
        print linea
\end{verbatim}
\afterverb
%
Abrimos el fichero, vamos recorriendo cada línea, y usamos la función {\tt search()},
de la librería de expresiones regulares, para
imprimir solamente aquellas líneas que contienen la cadena ``From:''. Este programa no usa la potencia
real de las expresiones regulares, ya que podríamos haber usado simplemente {\tt linea.find()} para
lograr el mismo resultado.
\index{cadena!find}

La potencia de las expresiones regulares llega cuando añadimos caracteres especiales a la cadena de
búsqueda, que nos permiten controlar con mayor precisión qué líneas coinciden con nuestro patrón.
Añadir estos caracteres especiales a nuestra expresión regular nos permite localizar patrones
complejos y realizar extracciones escribiendo muy poco código.

Por ejemplo, el carácter de intercalación (\verb"^") es usado en las expresiones
regulares para indicar ``el comienzo'' de una línea.
Podemos cambiar nuestro programa para que sólo localice
aquellas líneas en las cuales ``From:'' esté al principio:

\beforeverb
\begin{verbatim}
import re
manf = open('mbox-short.txt')
for linea in manf:
    linea = linea.rstrip()
    if re.search('^From:', linea) :
        print linea
\end{verbatim}
\afterverb
%
Ahora sólo coincidirán las líneas que {\em comiencen} por la cadena ``From:''. Todavía se trata
de un ejemplo de análisis muy sencillo, ya que podríamos haber hecho lo mismo con el método
{\tt startswith()} de la librería de cadenas. Pero sirve para introducir la noción de que las expresiones
regulares contienen caracteres de acción especiales que nos dan más control sobre lo que localizará
la expresión regular.
\index{cadena!startswith}

\section{Equivalencia de caracteres en expresiones regulares}

Hay varios caracteres especiales más que nos permiten construir expresiones regulares aún más potentes.
El más usado de ellos es el punto o parada completa, que equivale
a cualquier carácter.
\index{comodín}
\index{regex!comodín}

En el ejemplo siguiente, la expresión regular ``F..m:'' coincidirá con cualquiera de las cadenas
``From:'', ``Fxxm:'', ``F12m:'', or ``F!@m:'', ya que el carácter punto en la expresión regular
equivale a cualquier carácter.

\beforeverb
\begin{verbatim}
import re
manf = open('mbox-short.txt')
for linea in manf:
    linea = linea.rstrip()
    if re.search('^F..m:', linea) :
        print linea
\end{verbatim}
\afterverb
%
Esto resulta particularmente potente cuando se combina con la capacidad de indicar que un carácter
puede repetirse cualquier número de veces, usando los caracteres ``*'' o ``+'' en la expresión regular.
Estos caracteres especiales significan que en vez de coincidir un único carácter con la cadena buscada,
pueden coincidir cero-o-más caracteres (en el caso del asterisco), o uno-o-más caracteres
(en el caso del signo más).

Podemos restringir aún más las líneas que coincidirán con la búsqueda, usando un carácter {\bf comodín}
que se repita, como en el ejemplo siguiente:

\beforeverb
\begin{verbatim}
import re
manf = open('mbox-short.txt')
for linea in manf:
    linea = linea.rstrip()
    if re.search('^From:.+@', linea) :
        print linea
\end{verbatim}
\afterverb
%
La cadena buscada ``\verb"^"From:.+@'' encontrará todas las líneas que comienzan con ``From:'',
seguido por uno o más caracteres (``.+''), seguidos de un símbolo-arroba. De modo que la
línea siguiente sería localizada:

\beforeverb
\begin{alltt}
{\bf From:}\underline{ stephen.marquard}{\bf @}uct.ac.za
\end{alltt}
\afterverb

Puedes pensar en el comodín ``.+'' como una extensión que equivale a todos los caracteres
que están entre los dos-puntos y el símbolo arroba.

\beforeverb
\begin{alltt}
{\bf From:}\underline{.+}{\bf @}
\end{alltt}
\afterverb

Resulta útil pensar en los caracteres más y asterisco como ``empujadores''. Por ejemplo, en la cadena
siguiente la coincidencia con nuestra expresión llegaría hasta el último signo arroba de la línea,
ya que el ``.+'' empuja hacia fuera, como se muestra debajo:

\beforeverb
\begin{alltt}
{\bf From:}\underline{ stephen.marquard@uct.ac.za, csev@umich.edu, and cwen}{\bf @}iupui.edu
\end{alltt}
\afterverb

Es posible decirle a un asterisco o a un signo más que no sean ``codiciosos'', añadiendo
otro carácter. Mira la documentación detallada para obtener información sobre cómo desactivar
el comportamiento codicioso.
\index{codicioso}

\section{Extracción de datos usando expresiones regulares}

Si queremos extraer datos desde una cadena en Python, podemos usar el método {\tt findall()} para
obtener todas las subcadenas que coinciden con una expresión regular. Pongamos por caso que
queramos extraer todo aquello que se parezca a una dirección de e-mail de cualquier línea, independientemente
del formato de la misma. Por ejemplo, deseamos extraer las direcciones de e-mail de cada una de las
líneas siguientes:

\beforeverb
\begin{verbatim}
From stephen.marquard@uct.ac.za Sat Jan  5 09:14:16 2008
Return-Path: <postmaster@collab.sakaiproject.org>
          for <source@collab.sakaiproject.org>;
Received: (from apache@localhost)
Author: stephen.marquard@uct.ac.za
\end{verbatim}
\afterverb
%
No queremos escribir código para cada uno de los tipos de líneas, dividirlas y rebanarlas de forma diferente
en cada caso. El programa siguiente usa {\tt findall()} para localizar las líneas que contienen
direcciones de e-mail, y extraer una o más direcciones de cada una de ellas.
\index{findall}
\index{regex!findall}

\beforeverb
\begin{verbatim}
import re
s = 'Hello from csev@umich.edu to cwen@iupui.edu about the meeting @2PM'
lst = re.findall('\S+@\S+', s)
print lst
\end{verbatim}
\afterverb
%
El método {\tt findall()} busca la cadena que se le pasa como segundo argumento y en este caso devuelve
una lista de todas las cadenas que parecen direcciones de e-mail. Estamos usando una secuencia de dos
caracteres, que equivale a cualquier carácter distinto de un espacio en blanco ({\textbackslash}S). 

La salida del programa sería:

\beforeverb
\begin{verbatim}
['csev@umich.edu', 'cwen@iupui.edu']
\end{verbatim}
\afterverb
%
Traduciendo la expresión regular, estamos buscando subcadenas que tengan al menos un
carácter que no sea un espacio en blanco, seguido por un signo arroba, seguido por al menos un carácter más
que tampoco sea un espacio en blanco. El ``{\textbackslash}S+'' equivale a tantos caracteres
no-espacio-en-blanco como sea posible.

La expresión regular encontrará dos coincidencias (csev@umich.edu y cwen@iupui.edu), pero no
capturará la cadena ``@2PM'', ya que no hay ningún carácter distinto de espacio en blanco {\em antes} del
símbolo arroba. Podemos usar esta expresión regular en un programa para leer todas las líneas de un archivo
y mostrar en pantalla todo lo que se parezca a una dirección de correo electrónico: 

\beforeverb
\begin{verbatim}
import re
manf = open('mbox-short.txt')
for linea in manf:
    linea = linea.rstrip()
    x = re.findall('\S+@\S+', linea)
    if len(x) > 0 :
        print x
\end{verbatim}
\afterverb
%
Vamos leyendo cada línea y luego extraemos todas las subcadenas que coinciden con nuestra expresión regular.
Como {\tt findall()} devuelve una lista, simplemente comprobamos si el número de elementos en la lista
de retorno es mayor que cero, para mostrar sólo aquellas líneas en las cuales hemos encontrado al menos una subcadena
que parece una dirección de e-mail.

Si se hace funcionar el programa con {\tt mbox.txt}, obtendremos la siguiente salida:

\beforeverb
\begin{verbatim}
['wagnermr@iupui.edu']
['cwen@iupui.edu']
['<postmaster@collab.sakaiproject.org>']
['<200801032122.m03LMFo4005148@nakamura.uits.iupui.edu>']
['<source@collab.sakaiproject.org>;']
['<source@collab.sakaiproject.org>;']
['<source@collab.sakaiproject.org>;']
['apache@localhost)']
['source@collab.sakaiproject.org;']
\end{verbatim}
\afterverb
%
Algunas de nuestras direcciones de correo tienen caracteres incorrectos, como  ``\verb"<"'' o ``;'' al principio
o al final. Vamos a indicar que sólo estamos interesados en la porción de la cadena que comienza y
termina con una letra o un número.

Para lograrlo, usaremos otra característica de las expresiones regulares. Los corchetes se utilizan para indicar
un conjunto de varios caracteres aceptables que estamos dispuestos a considerar coincidencias. En cierto sentido,
el ``{\textbackslash}S'' ya está exigiendo que coincidan con el conjunto de ``caracteres que no son espacios en
blanco''. Ahora vamos a ser un poco más explícitos en cuanto a los caracteres con los que queremos que coincida.

He aquí nuestra nueva expresión regular:

\beforeverb
\begin{verbatim}
[a-zA-Z0-9]\S*@\S*[a-zA-Z]
\end{verbatim}
\afterverb
%
Esto se va volviendo un poco complicado y seguramente ya empiezas a ver por qué las expresiones regulares tienen
su propio lenguaje para ellas solas. Traduciendo esta expresión regular, estamos buscando subcadenas que
comiencen con una {\em única} letra (minúscula o mayúscula), o un número ``[a-zA-Z0-9]'', seguido por cero
o más caracteres no-en-blanco (``{\textbackslash}S*''), seguidos por un símbolo arroba, seguidos por cero
o más caracteres no-en-blanco (``{\textbackslash}S*''), seguidos por una letra mayúscula o minúscula.
Fíjate que hemos cambiado de ``+'' a ``*'' para indicar cero o más caracteres no-blancos, dado que ``[a-zA-Z0-9]''
ya es un carácter no-en-blanco. Recuerda que el ``*'' o ``+'' se aplica al carácter que queda inmediatamente
a la izquierda del más o del asterisco.

Si aplicamos esta expresión en nuestro programa, los datos quedan mucho más limpios:

\beforeverb
\begin{verbatim}
import re
manf = open('mbox-short.txt')
for linea in manf:
    linea = linea.rstrip()
    x = re.findall('[a-zA-Z0-9]\S*@\S*[a-zA-Z]', linea)
    if len(x) > 0 :
        print x
\end{verbatim}
\afterverb
%

\beforeverb
\begin{verbatim}
...
['wagnermr@iupui.edu']
['cwen@iupui.edu']
['postmaster@collab.sakaiproject.org']
['200801032122.m03LMFo4005148@nakamura.uits.iupui.edu']
['source@collab.sakaiproject.org']
['source@collab.sakaiproject.org']
['source@collab.sakaiproject.org']
['apache@localhost']
\end{verbatim}
\afterverb
%
Fíjate que en las líneas de ``source@collab.sakaiproject.org'', nuestra expresión regular
ha eliminado dos letras del final de la cadena (``\verb">";''). Esto se debe a que cuando
añadimos ``[a-zA-Z]'' al final de nuestra expresión regular, le estamos pidiendo que cualquier
cadena que el analizador de expresión regulares encuentre debe terminar con una letra. De modo que cuando ve el
``\verb">"'' después de``sakaiproject.org\verb">";'' simplemente se detiene en la última letra que
ha encontrado que ``coincide'' (es decir, la ``g'' en este caso).

Observa también que la salida de este programa consiste, para cada línea, en una lista de Python que contiene
una cadena como único elemento.

\section{Combinar búsqueda y extracción}

Si queremos encontrar números en líneas que comienzan con la cadena ``X-'', como en:

\beforeverb
\begin{verbatim}
X-DSPAM-Confidence: 0.8475
X-DSPAM-Probability: 0.0000  
\end{verbatim}
\afterverb
%
no querremos simplemente localizar cualquier número en punto flotante de cualquier línea. Querremos extraer
solamente los números de las líneas que tengan la sintaxis indicada arriba.

Podemos construir la siguiente expresión regular para elegir las líneas:

\beforeverb
\begin{verbatim}
^X-.*: [0-9.]+
\end{verbatim}
\afterverb
%
Traducido, lo que estamos diciendo es que queremos las líneas que comiencen con ``X-'', seguidas por cero o
más caracteres (``.*''), seguidas por dos-puntos (``:'') y luego un espacio. Después del espacio busca
uno o más caracteres que sean o bien dígitos (0-9) o puntos ``[0-9.]+''.
Fíjate que dentro de los corchetes, el punto coincide con un punto real (es decir, dentro de los corchetes
no actúa como comodín).

Se trata de una expresión muy rigurosa, que localizará bastante bien únicamente las líneas en las que
estamos interesados:

\beforeverb
\begin{verbatim}
import re
manf = open('mbox-short.txt')
for linea in manf:
    linea = linea.rstrip()
    if re.search('^X\S*: [0-9.]+', linea) :
        print linea
\end{verbatim}
\afterverb
%
Cuando ejecutemos el programa, veremos los datos correctamente filtrados
para mostrar sólo las líneas que estamos buscando.

\beforeverb
\begin{verbatim}
X-DSPAM-Confidence: 0.8475
X-DSPAM-Probability: 0.0000
X-DSPAM-Confidence: 0.6178
X-DSPAM-Probability: 0.0000
\end{verbatim}
\afterverb
%
Pero ahora debemos resolver el problema de la extracción de los números. A pesar de que resultaría
bastante sencillo usar {\tt split}, podemos usar otra características de las expresiones regulares para
buscar y analizar la línea al mismo tiempo.
\index{cadena!split}

Los paréntesis son también caracteres especiales en las expresiones regulares. Cuando se añaden paréntesis
a una expresión regular, éstos se ignoran a la hora de buscar coincidencias. Pero cuando se usa
{\tt findall()}, los paréntesis indican que a pesar de que se desea que la expresión completa coincida,
sólo se está interesado en extraer una cierta porción de la subcadena.
\index{regex!paréntesis}
\index{paréntesis!expresiones regulares}

Así que haremos el siguiente cambio en nuestro programa:

\beforeverb
\begin{verbatim}
import re
manf = open('mbox-short.txt')
for linea in manf:
    linea = linea.rstrip()
    x = re.findall('^X\S*: ([0-9.]+)', linea)
    if len(x) > 0 :
        print x
\end{verbatim}
\afterverb
%
En vez de llamar a {\tt search()}, añadimos paréntesis alrededor de la parte de la expresión regular
que representa el número en punto flotante, para indicar que queremos que {\tt findall()} sólo nos devuelva
la porción con el número en punto flotante de la cadena coincidente.

La salida de este programa es la siguiente:

\beforeverb
\begin{verbatim}
['0.8475']
['0.0000']
['0.6178']
['0.0000']
['0.6961']
['0.0000']
..
\end{verbatim}
\afterverb
%
Los número siguen estando en una lista y aún necesitan ser convertidos de cadenas a números en punto flotante, pero
hemos usado el poder de las expresiones regulares para realizar tanto la búsqueda como la extracción de información
que nos resulta interesante.

Como otro ejemplo más de esta técnica, si observas el archivo verás que hay un cierto número de líneas
con esta forma:

\beforeverb
\begin{verbatim}
Details: http://source.sakaiproject.org/viewsvn/?view=rev&rev=39772
\end{verbatim}
\afterverb
%
Si queremos extraer todos los números de revisión (los números enteros al final de esas líneas)
usando la misma técnica que en el caso anterior, podríamos escribir el programa siguiente:

\beforeverb
\begin{verbatim}
import re
manf = open('mbox-short.txt')
for linea in manf:
    linea = linea.rstrip()
    x = re.findall('^Details:.*rev=([0-9]+)', linea)
    if len(x) > 0:
        print x
\end{verbatim}
\afterverb
%
Traduciendo nuestra expresión regular, estamos buscando aquellas líneas que comiencen con ``Details:'',
seguido de cualquier número de caracteres (``*''), seguido por ``rev='', y luego por uno o más dígitos.
Queremos encontrar las líneas que coincidan con la expresión completa, pero deseamos extraer unicamente
el número entero al final de la línea, de modo que rodeamos con paréntesis ``[0-9]+''.

Cuando ejecutamos el programa, obtenemos la salida siguiente:

\beforeverb
\begin{verbatim}
['39772']
['39771']
['39770']
['39769']
...
\end{verbatim}
\afterverb
%
Recuerda que el ``[0-9]+'' es ``codicioso'', e intentará conseguir una cadena de dígitos tan larga
como sea posible antes de extraer esos dígitos. Este comportamiento ``codicioso'' es el motivo por el que obtenemos
los cinco dígitos de cada número. La expresión regular se expande en ambas direcciones hasta que
encuentra un no-dígito, o el comienzo o final de una línea.

Ahora ya podemos usar expresiones regulares para rehacer un ejercicio anterior del libro, en el cual estábamos
interesados en la hora de cada mensaje de e-mail. Buscamos líneas de la forma:

\beforeverb
\begin{verbatim}
From stephen.marquard@uct.ac.za Sat Jan  5 09:14:16 2008
\end{verbatim}
\afterverb
%
y queremos extraer la hora del día de cada linea. Anteriormente lo hicimos con dos llamadas a
{\tt split}. Primero dividíamos la línea en palabras y luego cogíamos la quinta palabra y la dividíamos
de nuevo usando el carácter dos-puntos para extraer los dos caracteres en los que estábamos interesados.

% Añadir una sección con la noción de código frágil
A pesar de que esto funciona, en realidad se está generando un código bastante frágil, que asume que todas las líneas
están perfectamente formateadas. Si quisieras añadir suficiente comprobación de errores (o un gran bloque try/except)
para asegurarte de que el programa nunca falle cuando se encuentre con líneas incorrectamente formateadas,
el código aumentaría en 10-15 líneas bastante difíciles de leer.

Podemos lograr lo mismo de un modo mucho más sencillo con la siguiente expresión regular:

\beforeverb
\begin{verbatim}
^From .* [0-9][0-9]:
\end{verbatim}
\afterverb
%
La traducción de esta expresión regular es que estamos buscando líneas que comiencen con ``From ''
(fíjate en el espacio), seguidas por cualquier cantidad de caracteres (``.*''), seguidos por un espacio,
seguidos por dos dígitos ``[0-9][0-9]'', seguidos por un carácter dos-puntos. Esta es la definición del tipo de
líneas que estamos buscando.

Para extraer sólo la hora usando {\tt findall()}, vamos a añadir paréntesis alrededor de los dos dígitos,
de este modo:

\beforeverb
\begin{verbatim}
^From .* ([0-9][0-9]):
\end{verbatim}
\afterverb
%
El programa quedaría entonces así:

\beforeverb
\begin{verbatim}
import re
manf = open('mbox-short.txt')
for linea in manf:
    linea = linea.rstrip()
    x = re.findall('^From .* ([0-9][0-9]):', linea)
    if len(x) > 0 : print x
\end{verbatim}
\afterverb
%
Cuando el programa se ejecuta, produce la siguiente salida:

\beforeverb
\begin{verbatim}
['09']
['18']
['16']
['15']
...
\end{verbatim}
\afterverb
%
\section{Escapado de caracteres}

Los caracteres especiales se utilizan en las expresiones regulares para localizar el principio o el final
de una línea, o también como comodines, así que necesitamos un modo de indicar cuándo esos caracteres son
``normales'' y lo queremos encontrar es el carácter real, como un signo de dolar o uno de intercalación.

Podemos indicar que deseamos que simplemente equivalga a un carácter poniendo delante de ese carácter
una barra invertida. A esto se le llama ``escapar'' el carácter. Por ejemplo, podemos encontrar
cantidades de dinero con la expresión regular siguiente:

\beforeverb
\begin{verbatim}
import re
x = 'Acabamos de recibir 10.00$ por las galletas.'
y = re.findall('[0-9.]+\$',x)
\end{verbatim}
\afterverb
%
Dado que hemos antepuesto una barra invertida al signo dólar, ahora equivaldrá al símbolo del dolar
en la cadena de entrada, en lugar de equivaler al ``final de la línea'', y el resto de la expresión
regular buscará uno o más dígitos o el carácter punto. {\em Nota:} Dentro de los corchetes,
los caracteres no son ``especiales''. De modo que cuando escribimos ``[0-9.]'', en realidad
significa dígitos o un punto. Fuera de los corchetes, un punto es el carácter ``comodín'' y
coincide con cualquier carácter. Dentro de los corchetes, el punto es simplemente un punto.

\section{Resumen}

Aunque sólo hemos arañado la superficie de las expresiones regulares, hemos aprendido un poco
acerca de su lenguaje. Hay cadenas de búsqueda con caracteres especiales en su interior que
comunican al sistema del expresiones regulares nuestros deseos acerca de qué queremos ``buscar''
y qué se extraerá de las cadenas que se localicen. Aquí tenemos algunos de esos caracteres especiales
y secuencias de caracteres:

\verb"^" \newline
Coincide con el principio de una línea.

\$ \newline
Coincide con el final de una línea.

. \newline
Coincide con cualquier carácter (un comodín).

{\textbackslash}s \newline
Coincide con un carácter espacio en blanco.

{\textbackslash}S \newline
Coincide con cualquier carácter que no sea un espacio en blanco (opuesto a {\textbackslash}s).

* \newline
Se aplica al carácter que le precede e indica que la búsqueda debe coincidir cero o más veces con él.

*? \newline
Se aplica al carácter que le precede e indica que la búsqueda debe coincidir cero o más veces con él
en ``modo no-codicioso''.

+ \newline
Se aplica al carácter que le precede e indica que la búsqueda debe coincidir una o más veces con él.

+? \newline
Se aplica al carácter que le precede e indica que la búsqueda debe coincidir una o más veces con él
en ``modo no-codicioso''.

[aeiou] \newline
Coincide con un único carácter siempre que ese carácter esté en el conjunto especificado. En este ejemplo,
deberían coincidir ``a'', ``e'', ``i'', ``o'', o ``u'', pero no los demás caracteres.

[a-z0-9] \newline
Se pueden especificar rangos de caracteres usando el guión. Este ejemplo indica un único carácter
que puede ser una letra minúscula o un dígito.

[\verb"^"A-Za-z] \newline
Cuando el primer carácter en la notación del conjunto es un símbolo de intercalación, se invierte la lógica.
En este ejemplo, la expresión equivale a un único carácter que sea cualquier cosa {\em excepto} una letra
mayúscula o minúscula.

( ) \newline
Cuando se añaden paréntesis a una expresión regular, éstos son ignorados durante la búsqueda,
pero permiten extraer un subconjunto particular de la cadena localizada en vez de la cadena
completa, cuando usamos {\tt findall()}.

{\textbackslash}b \newline
Coincide con la cadena vacía, pero sólo al principio o al final de una palabra.

{\textbackslash}B \newline
Coincide con la cadena vacía, pero no al principio o al final de una palabra.

{\textbackslash}d \newline
Coincide con cualquier dígito decimal, es equivalente al conjunto [0-9].

{\textbackslash}D \newline
Coincide con cualquier carácter que no sea un dígito; equivale al conjunto [\verb"^"0-9].

\section{Sección extra para usuarios de Unix}

El soporte para búsqueda de archivos usando expresiones regulares viene incluido dentro del sistema operativo Unix
desde los años 1960, y está disponible en casi todos los lenguajes de programación de una u otra forma.

\index{grep}
De hecho, existe un programa de línea de comandos integrado en Unix
llamado {\bf grep} ({\tt Generalized Regular Expression Parser} - Analizador Generalizado de Expresiones Regulares) que
hace casi lo mismo que hemos visto con {\tt search()} en los ejemplos de este capítulo. De modo que si tienes un
sistema Macintosh o Linux, puedes probar las siguientes órdenes en la ventana de línea de comandos:

\beforeverb
\begin{verbatim}
$ grep '^From:' mbox-short.txt
From: stephen.marquard@uct.ac.za
From: louis@media.berkeley.edu
From: zqian@umich.edu
From: rjlowe@iupui.edu
\end{verbatim}
\afterverb
%
Esto le dice a {\tt grep} que muestre las líneas que comienzan con la cadena ``From:'' del archivo
{\tt mbox-short.txt}. Si experimentas un poco con el comando {\tt grep} y lees su documentación,
encontrarás algunas sutiles diferencias entre el soporte de expresiones regulares en Python y el
de {\tt grep}. Por ejemplo, {\tt grep} no soporta el carácter equivalente a no-espacio-en-blanco,
``{\textbackslash}S'', de modo que hay que usar la notación bastante más compleja ``[\verb"^" ]'',
que simplemente significa que busque un carácter que sea cualquier cosa distinta a un espacio.

\section{Depuración}

Python tiene cierta documentación sencilla y rudimentaria que puede llegar a ser bastante útil si
necesitas un repaso rápido que active tu memoria acerca del nombre exacto de un método particular.
Esta documentación puede verse en el intérprete de Python en modo interactivo.

Puedes acceder al sistema de ayuda interactivo usando {\tt help()}.

\beforeverb
\begin{verbatim}
>>> help()

Welcome to Python 2.6!  This is the online help utility.

If this is your first time using Python, you should definitely check out
the tutorial on the Internet at http://docs.python.org/tutorial/.

Enter the name of any module, keyword, or topic to get help on writing
Python programs and using Python modules.  To quit this help utility and
return to the interpreter, just type "quit".

To get a list of available modules, keywords, or topics, type "modules",
"keywords", or "topics".  Each module also comes with a one-line summary
of what it does; to list the modules whose summaries contain a given word
such as "spam", type "modules spam".

help> modules
\end{verbatim}
\afterverb
%
Si sabes qué módulo quieres usar, puedes utilizar el comando {\tt dir()} para localizar los métodos del módulo, como
se muestra a continuación:

\beforeverb
\begin{verbatim}
>>> import re
>>> dir(re)
[.. 'compile', 'copy_reg', 'error', 'escape', 'findall', 
'finditer', 'match', 'purge', 'search', 'split', 'sre_compile', 
'sre_parse', 'sub', 'subn', 'sys', 'template']
\end{verbatim}
\afterverb
%
También puedes obtener un poco de documentación acerca de un método particular usando el comando dir.

\beforeverb
\begin{verbatim}
>>> help (re.search)
Help on function search in module re:

search(pattern, string, flags=0)
    Scan through string looking for a match to the pattern, returning
    a match object, or None if no match was found.
>>> 
\end{verbatim}
\afterverb
%
La documentación integrada no es muy extensa, pero puede resultar útil cuando tienes prisa,
o no tienes acceso a un navegador web o a un motor de búsqueda.

\section{Glosario}

\begin{description}

\item[código frágil:]
Código que funciona cuando los datos de entrada tienen un formato particular, pero es propenso a fallar
si hay alguna desviación del formato correcto. Llamamos a eso ``código frágil'',
porque ``se rompe'' con facilidad.

\item[coincidencia codiciosa:]
El concepto de que los caracteres ``+'' y ``*'' de una expresión regular se expanden hacia fuera para
capturar la cadena más larga posible.
\index{codicioso}
\index{coincidencia codiciosa}

\item[comodín:]
Un carácter especial que coincide con cualquier carácter. En las expresiones regulares, el carácter
comodín es el punto.
\index{comodín}

\item[expresión regular:]
Un lenguaje para expresar cadenas de búsqueda más complejas. Una expresión regular puede contener
caracteres especiales para indicar que una búsqueda sólo se realice en el principio o el final de una línea,
y muchas otras capacidades similares.

\item[grep:]
Un comando disponible en la mayoría de sistemas Unix que busca a través de archivos de texto, localizando líneas
que coincidan con una expresión regular. El nombre del comando significa "Generalized Regular Expression Parser"
(Analizador Generalizado de Expresiones Regulares).
\index{grep}

\end{description}

\section{Ejercicios}

\begin{ex}
Escribe un programa sencillo que simule la forma de operar del comando {\tt grep}
de Unix. Pide al usuario introducir una expresión regular y cuenta el número
de líneas que localiza a partir de ella:

\beforeverb
\begin{verbatim}
$ python grep.py
Introduzca una expresión regular: ^Author
mbox.txt tiene 1798 líneas que coinciden con ^Author

$ python grep.py
Introduzca una expresión regular: ^X-
mbox.txt tiene 14368 líneas que coinciden con ^X-

$ python grep.py
Introduzca una expresión regular: java$
mbox.txt tiene 4218 líneas que coinciden con java$
\end{verbatim}
\afterverb
%
\end{ex}

\begin{ex}
Escribe un programa para buscar líneas que tengan esta forma:

\verb"New Revision: 39772"

y extrae el número de cada una de esas líneas usando una expresión regular
y el método {\tt findall()}. Calcula la media y el total y
muestra al final la media obtenida.

\beforeverb
\begin{verbatim}
Introduzca fichero:mbox.txt 
38549.7949721

Introduzca fichero:mbox-short.txt
39756.9259259
\end{verbatim}
\afterverb
%

\end{ex}